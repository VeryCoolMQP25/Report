%========================%
%    Initial Settings    %
%========================%
%  ignore all lines up   %
%       to line 78       %
%========================%

\documentclass{article}
\usepackage[utf8]{inputenc}
\usepackage[english]{babel}
\usepackage{titlesec}
\usepackage{hanging}
\usepackage{indentfirst}
\usepackage{setspace}
\usepackage{float}
\usepackage{multirow}
\usepackage{mathrsfs}
\usepackage{caption}
\usepackage{tocbasic}
\usepackage[toc,page]{appendix}
\DeclareTOCStyleEntry[beforeskip=.1em plus 1pt, pagenumberformat=\textbf]{tocline}{section}
\usepackage{adjustbox}
\usepackage[english]{babel}
\setlength{\parindent}{4em}
\setlength{\parskip}{0.5em}
\usepackage{array}
\usepackage{hyperref}
\hypersetup{
    colorlinks=false,
    linkcolor=black,
    filecolor=black,
    urlcolor=black,
}
\newcommand{\MYhref}[3][blue]{\href{#2}{\color{#1}{#3}}}%
\urlstyle{same}
\usepackage[letterpaper, portrait, margin=1in]{geometry}
\usepackage{graphicx}
\graphicspath{ {images/} }
\usepackage{array}
\newcolumntype{L}[1]{>{\raggedright\let\newline\\\arraybackslash\hspace{0pt}}m{#1}}
\newcolumntype{C}[1]{>{\centering\let\newline\\\arraybackslash\hspace{0pt}}m{#1}}
\newcolumntype{R}[1]{>{\raggedleft\let\newline\\\arraybackslash\hspace{0pt}}m{#1}}

\titleclass{\subsubsubsection}{straight}[\subsection]

\newcounter{subsubsubsection}[subsubsection]
\renewcommand\thesubsubsubsection{\thesubsubsection.\arabic{subsubsubsection}}

\titleformat{\subsubsubsection}
  {\normalfont\normalsize\bfseries}{\thesubsubsubsection}{1em}{}
\titlespacing*{\subsubsubsection}
{0pt}{3.25ex plus 1ex minus .2ex}{1.5ex plus .2ex}

\makeatletter
\renewcommand\paragraph{\@startsection{paragraph}{5}{\z@}%
  {3.25ex \@plus1ex \@minus.2ex}%
  {-1em}%
  {\normalfont\normalsize\bfseries}}
\renewcommand\subparagraph{\@startsection{subparagraph}{6}{\parindent}%
  {3.25ex \@plus1ex \@minus .2ex}%
  {-1em}%
  {\normalfont\normalsize\bfseries}}
\def\toclevel@subsubsubsection{4}
\def\toclevel@paragraph{5}
\def\toclevel@paragraph{6}
\def\l@subsubsubsection{\@dottedtocline{4}{7em}{4em}}
\def\l@paragraph{\@dottedtocline{5}{10em}{5em}}
\def\l@subparagraph{\@dottedtocline{6}{14em}{6em}}
\makeatother

\setcounter{secnumdepth}{4}
\setcounter{tocdepth}{4}

\renewcommand{\contentsname}{Table of Contents}
\renewcommand{\listtablename}{Tables}
\renewcommand{\listfigurename}{Figures}

%========================%
% Beginning of Document  %
%========================%

\begin{document} 

%========================%
%     General Note       %
%========================%
%  Start new line: '\\'  %
%  Start new paragraph:  %
%         '\par'         %
%========================%

%========================%
%       Title Page       %
%========================%

\pagenumbering{gobble} % Disable page number on title page
\begin{center}
    \Huge{\textbf{Title of MQP}} \\ % Input title of MQP
    \vspace{10mm} % Add vertical space between sections
    \large{
    A Major Qualifying Project (MQP) Report \\
    Submitted to the Faculty of \\
    WORCESTER POLYTECHNIC INSTITUTE \\
    in partial fulfillment of the requirements \\
    for the Degree of Bachelor of Science in \\
    } % Do not edit this portion
    \Large{
    \vspace{5mm} % Add vertical space between sections
    First major, \\ % Input first major
    Second major \\ % Input second major
    \vspace{10mm} % Add vertical space between sections
    By: \\
    \vspace{5mm} % Add vertical space between sections
    Author 1 Name \\ % Input first author name
    Author 2 Name \\ % Input second author name
    \vspace {15mm} % Add vertical space between sections
    Project Advisors: \\ % Edit if only one advisor
    \vspace{5mm} % Add vertical space between sections
    Advisor 1 \\ % Input name of first advisor
    Advisor 2 \\  % Input name of second advisor
    \vspace {15mm} % Add vertical space between sections
    Sponsored By: \\ 
    \vspace{5mm} % Add vertical space between sections
    Project Sponsor Name \\ % Input project sponsor(s)
    \vspace {10mm} % Add vertical space between sections
    Date: April 2021 \\ % Input date of project submission
    }
    \vspace{10mm} % Add vertical space between sections
    \large{\textit{This report represents work of WPI undergraduate students submitted to the faculty as evidence of a degree requirement. WPI routinely publishes these reports on its website without editorial or peer review. For more information about the projects program at WPI, see \url{http://www.wpi.edu/Academics/Projects}.}} % Do not edit this portion
\end{center}

%========================%
%   Delete unnecessary   %
%  portions and adjust   %
%   vertical spaces in   %
%  lines 89-121 so that  %
%   the text fills the   %
%   entire title page    %
%========================%

\newpage % Start new page
\pagenumbering{roman} % Set page numbering to lower case Roman numerals (Use 'Roman' for capital Roman numerals)
\setcounter{page}{1} % Set page number to 1

\section*{Abstract} % Start the section 'Abstract'. Include the asterisks to keep section unnumbered and off of the table of contents

\noindent Type abstract here. % Use '\noindent' to remove indentation from the first paragraph of each section
\par Second paragraph of abstract % Use '\par' to start a new paragraph

\newpage % Start new page
\section*{Acknowledgements} % Start the section 'Acknowledgements'. Include the asterisks to keep section unnumbered and off of the table of contents

\noindent Begin list of acknowledgements
\begin{enumerate} % Start a numbered list
    \item First Acknowledgement % Name of first acknowledgement
    \item Second Acknowledgement % Add new item with '\item'
\end{enumerate}

% OR %

\begin{itemize} % Start a bulletted list
    \item First Acknowledgement % Name of first acknowledgement
    \item Second Acknowledgement % Add new item with '\item'
\end{itemize}

\newpage % Start new page 
\tableofcontents % Table of contents
\listoftables % List of tables
\listoffigures % List of figures

\newpage % Start new page
\begin{doublespacing} % Adjust spacing
\pagenumbering{arabic} % Set page numbering to Arabic numbers
\setcounter{page}{1} % Set page number to 1

\section{Introduction} % Start the section 'Introduction'. Do not include the asterisks to add section number and include in table of contents

\noindent Give brief introduction to project. To use sources listed in the bibliography, use \cite{knuth:1984}. The citations \cite{latex:companion} will appear in the references \cite{latex2e} in the order in which they appear in the text \cite{texbook_label}, regardless of alphabetical order or the order in which they are listed in the .bib file.

\subsection{Project Objectives} % Start the subsection 'Project Objectives', as a subsection of 'Introduction'

\noindent Give brief overview of project objectives

\subsubsection{Task 1} % Start the subsubsection 'Task 1', as a subsection of 'Project Objectives'

\noindent Describe the first task. 

\subsubsection{Task 2} % Start the subsubsection 'Task 2', as a subsection of 'Project Objectives'

\noindent Describe the second task. 

\subsubsection{Task 3} % Start the subsubsection 'Task 3', as a subsection of 'Project Objectives'

\noindent Describe the third task. 

\subsection{Background} % Start the subsection 'Background', as a subsection of 'Introduction'

\noindent Give brief overview of project background

\newpage % Start new page
\section{Section Title}

\noindent Begin new section

%========================%
%   Inputting a table    %
%========================%

\begin{table}[!tb] % Use '!' to override the LaTeX table placement, 't' to place the table at the top of the page, and 'b' to place the table at the bottom of the page if the top does not work
    \begin{center}
        \begin{tabular}{ |c|c|c| } 
            \hline
            cell1 & cell2 & cell3 \\ 
            cell4 & cell5 & cell6 \\ 
            cell7 & cell8 & cell9 \\ 
            \hline
        \end{tabular}
        \caption{Caption of table} % Automatically adds a table number which updates as more tables are added to the paper
        \label{table:label_of_table} % Used in the paper to reference the table and automatically includes the table number in the in-text reference.
    \end{center}
\end{table}

\par To reference a table, you would simply say Table \ref{table:label_of_table}. By using the table label, the reference updates automatically as the paper is edited and more tables are added.

% For more information, go to https://www.overleaf.com/learn/latex/tables

\section{Section Title}

\noindent Begin new section

%========================%
%   Inputting an image   %
%========================%

\par To add an image to the document, you first have to upload it to the file. To do so, click the "Upload" button on the top left of the screen (just below the "Menu" button, pictured in the figure below) and upload the image from your computer. 

\begin{figure}[h] % Use 'h' to tell overleaf to place the image approximately 'here', meaning that you want it to be somewhere near where you are putting it in the text but will allow LaTeX to move it so that the overall format is best
    \centering
    \includegraphics[width=0.5\textwidth]{Images/upload_image.JPG} % Set the image width relative to the with of the text on the page and input the name of the image that is being placed.
    \caption{Figure Caption} % Automatically adds a figure number which updates as more figures are added to the paper
    \label{fig:label_of_figure} % Used in the paper to reference the figure and automatically includes the figure number in the in-text reference.
\end{figure}

\par To reference a figure, you would simply say Figure \ref{fig:label_of_figure}. By using the figure label, the reference updates automatically as the paper is edited and more figures are added.

% For more information, go to https://www.overleaf.com/learn/latex/Inserting_Images

\section{Section Title}

\noindent Begin new section

%========================%
% Inputting an equation  %
%========================%

% Use '\begin{equation}' to input a numbered equation. Used when equations will be reference in-text. Unlike tables, this equation will appear exactly where you put it in the text.

\begin{equation} 
    y=mx+b
    \label{eq:equation_label}
\end{equation}

\par Once the equation is inputted, you can reference it similarly to tables using 'Equation \ref{eq:equation_label}'. Just like tables, the equation number update automatically when the paper is edited and new equations are added. 

\par If you want to include an equation that is not numbered, use the form $$\frac{-b\pm \sqrt{b^{2}-4ac}}{2a}$$ and the equation will appear exactly where you put it in the sentence, in it's own line. 

\par If you want to include an equation in a sentence without putting it in its own line, use the LaTeX 'math mode' using the form $x_{1}^{2}+x_{2}^{2}=r^{2}$. 

% For more information and different methods, go to https://www.overleaf.com/learn/latex/mathematical_expressions

\section{Conclusion}

\noindent Input your conclusion here. 

\par Notes: 
\begin{enumerate}
    \item The table of contents is set up with hyperlinks, and you can get to any section of the paper by clicking the section in the table of contents. Same for tables and figures. 
    \item You can input as many sources as you want into the .bib file. They will only appear in the 'References' section if they are cited in text. The 'References' section is organized in the order of which the sources are cited in text. The in text citations are also hyperlinked, and clicking one will bring you to that source in the 'References' section. 
    \item If you ever don't know how to do something, you can almost always find it on \url{https://www.overleaf.com/learn} or just by Googling. 
    \item The paper can be split into sections, subsections, and subsubsections, and subsubsubsections. These can be inputted in any part of the main body of the text, from the abstract to the conclusion. Adding an asterisks in the command line will prevent the section from getting a section number and appearing in the table of contents.
    \item Appendices are referenced the same as anything else, using Appendix \ref{appendix:appendix_a_label} and Appendix \ref{appendix:appendix_b_label}.
\end{enumerate}

\newpage % Start new page
\begin{appendices}

\section{Appendix A Title}
\label{appendix:appendix_a_label}

Input materials for Appendix A. Works the same as regular text, just do not include captions or labels on any tables or figures. Appendices can be referenced in text the same way you reference figures or tables, using the label. 

\newpage

\section{Appendix B Title}
\label{appendix:appendix_b_label}

Input materials for Appendix B

\end{appendices}

\newpage % Start new page
\end{doublespacing} % Return to single spacing
\addcontentsline{toc}{section}{References} % Add the 'References' section to the table of contents
\bibliographystyle{ieeetr} % Set the bibliography style
\bibliography{bibliography.bib} % Generate a bibliography from the .bib file with all of the references
\end{document}
